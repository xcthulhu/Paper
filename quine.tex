\intodo[caption={Rewrite section title name?}]{This section presents
  Quine's traditional interpretation of his criterium and
  counter-examples refuting that perspective.  But this paper doesn't actually
  refute Quine's criterium (ie, ``To be is to be the value of a
  variable''), because ultimately you will salvage this idea with
substituional quantifiers.}

In this section I present Quine's traditional interpretation of his
criteria for ontological commitment, and provide examples of sentences
it has difficulty analyzing.\ftfy{I added an introductory sentence to
  this section summarizing what you do in it.  Maybe elaborate?}  
\intodo[caption={Return to these counter-examples later?}]{After presenting the counter examples in this section, and the
flaws in Quines' analysis using traditional model theory, you might
want to return to the examples in this section to show how
the substitutional quantifier provides the appropriate semantics.}
\otodo[caption={Theories = collection of sentences?  What about
  deduction?}]{Aren't theories closed under deduction?  You talk about
  ``deriving'' things from them on the next page.  Also, ``theories''
  are a proof theoretic concept; your paper is about semantics, however.
  It may be prudent to caste your discussion in terms in of
  statements ``holding'' under a theory rather than being derived.} 
Quine's criteria of ontological commitment %will be understood as a
is a methodological claim.  It asserts the beings that must exist as a
consequence of a theory being true.  
A theory will be understood as a collection of
sentences. There are objections that can be raised against this
representation.  For example, one ought not suppose that the theory of
relativity is tied to any particular sentential formulation. Such
arguments will be ignored in the present work. Perhaps theories are in
the end more general entities, but this question is largely irrelevant
to the present concern with understanding whether it is reasonable to
logically question the ontological commitments of the sentences to
which an individual adheres.
\delete[caption={Delete this whole paragraph?}]{Defending a standard, 
logic based understanding of theories like this is not very relevant
to your paper.  Delete this whole paragraph?}

Quine claims the following as ``a standard whereby to decide what
ontology a given theory or form of discourse is committed to'':
\begin{te}
A theory is committed to those and only those entities to which the
bound variables of the theory must be capable of referring in order
that the affirmations in the theory be true \cite[p.13]{quine}.\\
\label{ontcommit}
\end{te}

Quine takes sentences such as (\cex) ``there is a dog''\label{cex:dog} to be
paradigmatic cases of ontological commitment.  This is intuitively
plausible.  If someone posits 
%the audacious claim that 
\eqref{cex:dog}, 
%it seems natural to take 
they are ontologically bound to the existence of something which is a dog.
%the speaker to some thing: a dog. It is also evident that 
The standard semantics of the
existential quantifier can represent this sentence and its appropriate
truth conditions in terms of dogs as `$\exists xDog(x)$'. 

%It is not intuitively clear, though, that this 
Logical translation does not
always clarify the ontological commitments at play, however.
Quine's analysis does not deal with figurative speech elegantly.
%has contributed to or clarified (a)'s utterer's commitment to a dog. 
%It seems that the burden of ontological commitment already resided in
%my understanding of the concept expressed by `dog'.  
Consider the claim that (\cex) ``there is a bee in her
bonnet''\label{cex:bee}. 
Most English-speaking individuals
%, inferring the speaker's intent and 
%understanding the idiomatic proposition expressed by this sentence, 
would take this utterance to be a 
%perhaps true 
claim about some woman's obsessiveness.
They would not consider the literal
existence of a bee in
anyone's bonnet.  In this way \eqref{cex:bee} is interpreted differently than \eqref{cex:dog},
even though it has he same form.  
Quine %takes it for granted that intentionally
grants that figurative speech is not ontologically binding, 
%but also supposes that the fanciful utterance in question must be
%taken 
by dismissing such sentences as false\footnote{
  ``One way in which a man may fail to share the ontological
  commitments of his discourse is \ldots by taking an attitude of
  frivolity.  The parent who tells the Cinderella story is no more
  committed to admitting a fairy godmother and a pumpkin coach into
  his own ontology than to admitting the story as true''
  \cite[p.103]{quine}.}. This is heavy handed and awkward.  
It is intuitively possible that \eqref{cex:bee} could communicate a
true proposition, while remaining ontologically unburdened with
anything but the existance of some girl.  This contradicts Quine's interpretation of his
ontological criteria. \ftfy{I heavily editted this, because it
  used a lot of indirect language}

In the above cases ontological commitment 
%appears to be largely
is determined by the intentions of a speaker and the meaning of the
phrases in question.  \delete[caption={Delete paragraph with run-on sentence?}]{This is a rather long run-one sentence in a
  paragraph that isn't doing much work.  Delete all of this?} 
Though there is no reason yet to suppose that
intentionally figurative language exists in one's best theory of the
world, it demonstrates a certain intuition---that language is used
in many different ways and first-order logic is incapable of
distinguishing between them---that could incline one towards a
rejection of the Quinean criteria of ontological commitment.
\intodo[caption={\emph{But you don't really reject Quine's
    criterium!}}]{\emph{But you don't really reject Quine's
    criterium!} 
You just reject Quine's traditional first order logic
  interpretation of his criterium.}

\otodo[caption={Stay away from worrying about ``derivation'' 
in a semantics paper}]{This paragraph begins by assuming theories involve derivation;
  shouldn't you be explicit about this?  Why not just say ``Suppose
  that if `there is an $x$' \textbf{holds} in a given theory then
  \ldots''?  That way you can stick to semantics which is all this
  paper is about.
  Also, avoid unecessary words like ``though''.}
Let us suppose, though, that if one can derive a literal statement of
the form
\reorganize[caption={Put pithy summary of Quine's interpretation at
  the beginning of the section}]{Why not say this pithy summary at the
  beginning of the section, and then proceed to hammer it with your counter-examples?}
``there is an $x$'' from a given theory then the theory is committed to an appropriate $x$.
As indicated in (\ref{arg1}), the
typical manner of deriving such a statement when it is not explicitly
present in a theory is through translation into a first-order language
with standard semantics.  That is, one takes a claim such as (\cex)
``Pegasus is a winged horse''\label{cex:pegasus} and translates it into a closed formula
of first-order logic such as `$WingedHorse(pegasus)$' from which, via
existential generalization, one can derive the quantified expression
`$\exists xWingedHorse(x)$' from which it follows, based on the
Quinean understanding of the existential quantifier, that (\cex) ``there
exists a winged horse.''\label{cex:winghorse}
\rewrite[caption={Confusing paragraphs}]{Try to be more up front that 
``Pegasus is a winged horse'' is yet another counter-example, which
exposes the problem with the traditional Quinian criterium as a
problem with the semantics of first order logic.}
\intodo{Try to be clearer that Quine's first order logic based ontology
  is overly dimissive of \eqref{cex:winghorse}, just like the bee-bonnet sentence.}

While \eqref{cex:pegasus} is intuitively true, \eqref{cex:winghorse}
is intuitively false.  There are
two standard responses.
\reorganize{You should be up-front that you
  are not entertaining Meinongian ontologies. Put this up at the
  beginning of the section or in the introduction.}
  First, one can follow Meinong and accept
abstract or non-existent entities into their domain of discourse and
thereby the world.  For those with nominalist leanings, this move is
unintelligible.  Second, one can follow Quine and reject the truth of
\eqref{cex:pegasus} on the grounds that no entity exists that satisfies `$\exists
xWingedHorse(x)$' and hence both \eqref{cex:pegasus} and \eqref{cex:winghorse} must be strictly speaking
false.\reorganize{You need to say this up front!} If one refuses to countenance abstract objects but insists
that \eqref{cex:pegasus} is true and \eqref{cex:winghorse} is false, some step of the preceding
derivation must be rejected.

Another counter example to Quine's reading of his criterium in first
order logic is Marcus' (\cex) ``there is a statue of
Venus in the Louvre''\label{cex:venus} \cite{marcus72}.  \rewrite{Assume
Quine is a nominalist, and write that this is the obvious twist he
would make to interpret Marcus}For the
nominalist, \eqref{cex:venus} must
be paraphrased to ``there is a statue named `Venus' in the Louvre"
which suffers from a lack of inferences I take to be valid (``there is
a statue of a Roman deity in the Louvre'') and a distortion of the
proposition seemingly expressed by \eqref{cex:venus}: that there is a statue
\emph{of} Venus in the Louvre.
\reorganize{Maybe you could put all of your counter examples in one
  block, up front, and then devote one paragraph to each?}

I take the denial of these sentences' truth to be unsatisfactory.
It omits of facts I find obvious.  
\delete{You don't need to worry about Meinong here, so you should
  delete this.} The acceptance of abstract entities of whatever
caliber, though, is equally inexplicable.  
No matter one's account of such non-existent objects, this notion will 
constitute a more significant conflict with common sense than an 
appeal to the cultural
(or fictional) conventions one employs and reasons about in day-to-day
discourse. Accounting for the truth of sentences like \eqref{cex:pegasus} and \eqref{cex:venus} as
wholly conventional (and hence the ontological demands of such
sentences as nonexistent) is significantly more intuitive than the
alternative objectual interpretation.
\otodo{This is the only important part of this paragraph}
  That is to say, I take the
specification of a domain $D$ for such discourse, especially in the
murky conceptual territories in which ontological disputes arise, to
be more mysterious than the acceptance of a fictional or conventional
provision of truth values for sentences.
\intodo[caption={Ignore your positive view when presenting your
  negative view}]{Don't worry about sketching your \emph{positive}
  view in this section too much.  This section is all about your
  \emph{negative} view, and I would end with ``But even though it's
  all broke, we can fix it!'' and say no more}

\rewrite[caption={Weak bridge paragraph}]{This is a weak bridge
  paragraph.  Also, don't begin paragraphs with ``this''}This is not a move towards ontological relativity (that is, I do not
aim to advocate that what there is is a matter of convention), but
rather the claim that the truth of a sentence need not indicate
anything about the world but convention or, rather, that there is a
distinction between a theory of truth and the theory of reference one
needs to speak of ontological commitment.  The substitutional
quantifier can help to formalize the ambiguities at play here.

%%% Local Variables: 
%%% mode: latex
%%% TeX-master: "substitutionalquantification"
%%% End: 
