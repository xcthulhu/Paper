\rewrite{This isn't a good summary of what you are doing in this
  section.}
The most pressing concerns are how to account for the truth of atomic
sentences without recourse to an objectual domain, how to understand
the relation between the objectual and substitutional quantifiers, and
how to conceive the \emph{meaning} of a substitutionally quantified
phrase.

An obvious manner of providing the truth of atomic sentences is to
take the class of parameters, $C$, to contain properly referential
names and account for their truth or falsity by providing a domain of
discourse $D$ and a mapping $F$ from $C$ to $D$ and from n-ary
relations to the power set of $D^n$. A substitutional model $M$ can
now be defined such that $M(R^n_i(a_1, \ldots, a_n)) = T$ iff $\{F(a_1),
\ldots, F(a_n)\} \in F(R^n_i)$.  Here an atomic formula of a
substitutional language $L$ is true in a model if and only if it
contains a referent within the domain of discourse.  Such an
interpretation of the substitutional quantifier still defines the
truth of a first-order quantified sentence `$\Sigma xA$' in terms of
$A$'s truth when an atomic sentence is substituted for the variable
$x$ therein, but the truth of those substituents is dependent on
reference and hence any supposed ontological innocence is illusory.

This interpretation of a substitutional language is intelligible, but
clearly contrary to the metaphysical aims of the present paper.  It
does serve the purpose, though, of showing how substitutional
quantification can, in certain spheres of discourse, account for
ontological commitments.  That is, where the subject matter is taken
to be clearly defined (as is, one should note, almost never the case
in metaphysical disputes) and, as Marcus notes \cite{marcus72}, we can
be taken as already ontologically committed, then it is permissible to
see quantification as reinforcing such commitments by allowing the
substitution of referring names. This could be taken as a reversal of
the Quinean project on another front, with the bearer of ontological
burden reverting to the name.  For the adherent of the causal theory
of names, such a position might be appealing: ontological disputes
could become the consideration of a name's causal history in the
attempt to see whether it designates at all.

At the other extreme, a substitutional model could be understood
entirely without reference in terms of the conventions of a linguistic
community.  Such an interpretation could understand the substitutional
model to assign truth values to atomic sentences based on explicit
conventions of discourse or the specifications of a fictional corpus.
The parametric extension class, following Bonevac, can be seen to
contain different consistent frameworks or discourses and to detail
the language game relative to which we admit or reject candidates for
extension of our base assignment.  Bonevac uses such an
interpretation to account for the truth of mathematical statements
since, he argues, ``learning mathematics involves learning the rules
of mathematical discourse, and these rules allow existence assertions
on the basis of consistency alone" \cite[p.650]{bonevac84}.

Neither of these extremes seem \emph{philosophically} intelligible
given the assumption that we use language both to refer to the world
(such as should at some level involve reference) and to make
conventionally grounded statements (such as I cannot understand to
involve reference).  Kripke \cite{kripke} has demonstrated the
consistency of a logic that includes both objectual and substitutional
quantifiers that bind different kinds of variables. This formalism
would provide a means of reconciling these two intuitions.
Alternatively, the base parameters can be taken as standard
referential names, but extensions of this domain will be allowed
relative to convention or fiction.  For example, Greek mythology would
constitute an admissible extension that would allow one to make
certain true statements about Pegasus and Minotaurs without referring.

Either approach supports my underlying philosophical intuition: true
sentences can involve reference to the world or not and given some
sentence of natural language $s$ it is often ambiguous in which class
$s$ ought to fall.  In providing the logical form of $s$ with respect
to Kripke's formalism there would remain a substantive question of
what kind of variable, substitutional or objectual, to use in its
translation; in analyzing the logical form of a sentence according to
the second formalism, there would be, in metaphysical contexts,
the difficult question as to the status of a given name and whether it
refers.

Though both formalisms lose the austerity and elegance of the standard
semantics for first-order logic, I maintain that there is a genuine
problem of the truth of sentences such as those involving Pegasus that
is unresolved using the standard interpretation of first-order logic.
Logical simplicity does not appear an adequate compensation for this
loss.

Another objection has been raised by Van Inwagen \cite{inwagen} in
which he argues that the meaning of a substitutionally quantified
phrase is incomprehensible.  The proposition expressed by `$\exists
xDog(x)$' is clear: there exists a dog. Yet while the truth conditions
of the expression `$\Sigma xDog(x)$' are known, the \emph{meaning} of
this phrase is not unless it is the metalinguistic proposition that
there is some true substitution instance of `$Dog(x)$'.  Certainly this
is not the meaning of ``there are dogs'', though.  For example, one
could consent to ``there are dogs'' without holding beliefs about
sentences.

Scott Soames \cite[p.91]{soames} responds that if Van Inwagen's
complaint is legitimate, how are we to subsequently understand the
meaning of $A \& B$?  Since the truth of conjunction is defined in
terms of $A$'s being true and $B$'s being true, it seems to follow
that the meaning of `Dog(fido) \& Dog(lassy)' is a metalinguistic
proposition about the truth of `Dog(fido)' and `Dog(lassy)', but
this is surely not be the case.

This response hints at the fundamental problem of Van Inwagen's
critique: it seems to assume an identity between a theory of truth and
a theory of meaning. I have already claimed that sentences such as
``there are dogs'' can be taken as the clearest cases of ontological
commitment.  Thus, I do not find it problematic to claim that the
proposition expressed by (e) `$\Sigma xDog(x)$', if it is not already
a mistake to speak of the proposition expressed by a logical formula,
is the same as that expressed by (f) `$\exists xDog(x)$' which is the
same as that expressed by ``there are dogs''.  The difference remains
in how we account for the truth conditions of these sentences.  The
truth of (f) will be defined directly in terms of the dogs in the
world; the truth of (e) will be defined in terms of there being a name
$n$ such that `$Dog(n)$' is true and, as a contingent fact about the
meaning of `Dog', it will be the case that $n$ designates a dog.  In
following this line of argument, then, the only necessary constraints
on a theory of meaning are that meanings cannot be defined in terms of
reference and that a theory of meaning cannot be a theory of
truth. This goes against some traditional threads in the philosophy of
language but can perhaps find company with those who insist on an
external and conventional or normative account of meaning.

If no substantial technical difficulties are found, substitutional
quantification so defined provides a more coherent account of
fictional and conventional discourse than that provided by the use of
quantification over abstract entities, thereby resolving a non-trivial
philosophical problem. Some trade-offs for accepting this
philosophical use of the formalism are the constraints imposed on
one's theory of meaning and added logical complexity, but I contend
that this is a fair trade-off. 

Yet, that truths in everyday discourse can be grounded in convention
or fiction is relatively uncontroversial. The aspiring ontologist can
at this point still rebut that nothing demonstrated speaks to the
ontological commitments of the scientific discourse to be found in our
best theories of the world which, the ontologist hopes, does not
similarly make recourse to such non-committal figurative language.


%%% Local Variables: 
%%% mode: latex
%%% TeX-master: "substitutionalquantification"
%%% End: 
