Let a language $L$ consist of a countable set $V_L$ of individual
variables represented by `$x_1$', `$x_2$', \ldots, `$x_n$'; a
countable set $C_L$ of constants, represented by `$a_1$', `$a_2$',
\ldots, `$a_n$'; a finite or countable set $P_L$ of n-ary predicates for
each n, denoted `$R^n_1$', `$R^n_2$', \ldots, `$R^n_m$'; and the
logical symbols `$\neg$', `$\&$', `$\exists$', `$($', and `$)$'.

\error[caption={You
  are mistaken about the semenatics of relations in first order
  logic}]{Throughout your paper, you always talk about the semantics of
relations in terms of sets of n elements.  This is not right.  To see
this, assume
$\phi(R) = \{\{a,b\}\}$. Then $\mathfrak{M}\models R(x,y)$ if and 
only if $\mathfrak{M}\models R(y,x)$; in fact under this definition,
all relations are symmetric}
Recall the standard semantics for first-order logic.  A model $M$ of
$L$ is an ordered pair $<D,\phi>$ where $D$ is a non-empty set
constituting the domain of $M$ and $\phi$ is a maps elements
of $C_L$ to $D$ and n-ary predicates in $P_L$ to $D^n$,
that is the set of n-tuples of elements in $D$.  Let $V_M$ be
recursively defined as a valuation function on $M$ such that:

\begin{te}
  $V_M(R^n_i(b_1,\ldots ,b_n)) = T$ iff $< \phi(b_1),\ldots ,\phi(b_n) > \in
  \phi(R^n_i)$
\label{existclause}
\end{te}
\begin{te}
  $V_M(\neg A) = T$ iff $V_M(A) = F$
\end{te}
\begin{te}
  $V_M(A \& B) = T$ iff $V_M(A) = V_M(B) = T$
\end{te}\begin{te}
  $V_M(\exists xA) = T$ iff $V_{M'}(A[b/x]) = T$ for some
  parameter $b$ and some $M'$ differing from $M$ in at most what it
  assigns to $b$.\\
\label{quantifier}
\end{te}

It is (\ref{existclause}) that shows the referential nature of these
semantics: `$R(a,b)$' is true in a model only if the pair containing
the extension of `$a$' and the extension of `$b$', a pair of objects
from the domain $D$, is a member of the extension of
`$R$'. (\ref{quantifier}) in turn tells us that `$\exists xF(x)$' is true
just in case there is some element $a \in D$ such that if $\phi(x) =
{a}$, $F(x)$ is true.  The truth of a quantified sentence is therefore
dependent on its satisfaction by objects in the domain of discourse $D$.
\intodo{You need a citation for this.  I would say use Enderton, but
  you aren't following his definition\ldots}

A na\"ive account of substitutional semantics for first-order logic
requires several modifications to the above. So as to avoid notational
confusion, we will use the symbol $\Sigma$ to denote the
substitutional equivalent of the existential quantifier.  The formal
representation of na\"ive substitutional quantification constitutes the
following replacements for (\ref{existclause}) and (\ref{quantifier}):

\begin{te} 
$V_M(R^n_i(b_1,\ldots ,b_n)) = T$ iff $M(R^n_i(b_1,\ldots ,b_n)) = T$
\end{te}

\begin{te}
$V_M(\Sigma xA) = T$ iff $V_{M'}(A[b/x]) = T$ for some parameter $b
\in C_L$.\\
\end{te} 
For a substitutional language $L$, then, a model $M$ is defined
merely as a mapping from the closed, atomic formula of $L$ to
$\{T,F\}$.  Note that this posits the truth of an existentially
quantified formula `$\Sigma xF(x)$' as dependent only on a substitution
instance $a$ in $L$ such that $M(F(a)) = T$; there is no talk of a
domain $D$.

These na\"ive semantics, though, suffers from several notable
deficiencies.  First, the logic so described is neither compact nor
strongly complete.
\rewrite{A lot of philosophical logic isn't compact or strongly complete.
  Specifically, hybrid logic, epistemic logic, PDL, $\mu$-calculus,
  provability logic, counterfactual logic with the limit assumption,
  and logics from my own research all fail to have
  compactness or strong completeness.  These formal properties are
  fragile and break easily. And frankly, the more elaborate semantics
  you entertain are harder to understand than this. }
  If the substitutional quantifier is to be
supported for its pragmatic superiority to the objectual quantifier,
the former ought to share most of the latter's familiar logical
properties.
\intodo{A lot of logicians don't care about these results at all.
  Plenty of people use Higher Order Logic (like the Isabelle/HOL people)
  even though it's not even weakly complete.  Restricting yourself
  to only consider strongly complete, compact logics is really rather
  prohibitive.}

Second, such an interpretation is limited only to those terms
contained within the substitution class of parameters $C$ and thus,
since it cannot account for objects not already named in a language,
cannot account for much of natural discourse.  Na\"ive substitutional
quantifier semantics can only discuss domains that are countable and
named. 
%While we are interested in this quantifier for its ability to move us
%past logical questions of ontological obligation, it 
Since it is desirable to model uncountable domains commonly found in
science and mathematics, richer substitutional semantics desirable.
\otodo{If you are going to worry about expressiveness, you
  should mention at some point that FOL is embeddable in the logic for
  substitutional semantics (maybe you could sketch a proof in an appendix?)}
 The substitutional
quantifier must be at least as expressive as the standard existential
quantifier.

\reorganize{This paragraph is confusing.}Marcus \cite{marcus95} suggests that, if this second objection is legitimate,
first-order logic with objectual quantification does not fare much
better given the downward L\"owenheim-Skolem theorem.  Recall that if a theory
$T$ in a first-order language is satisfiable by an uncountable domain
then there exists a model $M'$ with a countable domain $D'$ that also
satisfies the sentences of $T$.  This result applies to the
first-order theory of real analysis that one would assume to be
satisfiable only by a domain with cardinality at least equal to that
of the real numbers. \rewrite{Real analysis isn't a first order
  theory.  You cannot express the least upper bound principle in first
  order logic.}
Consider now this model $M'$.  
\delete{It would be easier to delete this paragraph than rewrite it,
  since it isn't particularly pertinent to your paper}
Since $D'$ is
enumerable, a name can be provided for every object contained therein.
Therefore, if we let the parameters of a substitutional language $L$
be the names of objects in $D'$, we can construct a substitutional
model for $L$ according to $M'$ that satisfies $T$.
\intodo{Most model theorists don't restrict themselves to countable
  languages.  Theorems like Henkin's strong completeness, compactness,
  the L\"owenheim-Skolem theorems, can all be generalized to higher
  cardinalities using transfinite induction and lots of applications
  of the axiom of choice.  So really, you can just say ``There is some
substitutional language $L$ that models the first order theory of
$\mathbb{R}$'' without making smaller but elementarily equivalent 
models or anything like that.}

Bonevac \cite{bonevac84} addresses the above objections by proposing 
more powerful semantics for the substitutional quantifier.  A
parametric extension of a language $L$ is a language $L'$ defined to
be identical to $L$ except that $C_L \subset C_{L'}$ and $C_{L'}$ is
enumerable.  A model $M'$ for a language $L'$ is said to
parametrically extend a model $M$ for a language $L$ just in case $L'$
parametrically extends $L$ and for all atomic sentences $s$ of $L$,
$M'(s) = M(s)$; that is, $M'$ and $M$ agree on all of the atomic
sentences of $L$.  
%Denote such a restriction of a function $F$ to
%a domain $D$ $F |_D$.

\reorganize{Maybe this difficulty should just be in a footnote?}Na\"ive parametric extensional semantics for a language $L$ would say
that a well-formed formula $f$ is true in a model $M$ just in case
there is a parametric extension of $M$ according to which $f$ is
true. This is problematic since the language $L$ and
model $M$ could be modified without restriction; that is, such a
semantics seems to let one quantify over all logically possible
entities.  \delete{Delete this paragraph because it breaks up your
  explanation of the semantics and therefore makes them harder to
  understand} A manner of distinguishing between relevant and irrelevant
extensions of $L$ is necessary if this modified substitutional
quantifier is to be able to account for non-arbitrary scientific
discourse.

%Let a parametric extensional model, then, consist of a standard
%substitutional model coupled with a class of admissible parametric
%extensions; that is, 
A parametric model $M$ of $L$ is an ordered pair
$<\alpha, B>$ such that $\alpha$ is a function from the atomic formula
of $L$ into truth values and $B$ is a class of standard models that
parametrically extend $\alpha$.  
%While each
%parametric extension to $L$ has cardinality at most $\aleph_0$, the class $B$ could contain uncountably many functions
%designating admissible extensions.
\intodo[caption={Rewrite semantics to be more like modal logic?}]{If you wrote the semantics like
$\mathfrak{B},\alpha \models \phi$, it would be clearer to a casual
reader that this is a modal theory of quantification.}
\intodo[caption={Notes on intuitionistic logic}]{These semantics are extremely close to the semantics for
  intuitionistic propositional logic.  `$<M,L>$ is parametrically
  extended by $<M',L'>$' is the accessibility relation, and it is
  reflexive and transitive.  The fact that you enforce that when
  $<M,L> \leq <M',L'>$ then $M|_L = M'|_L$ is a form of
  \emph{persistance}, which is the character of intuitionistic
  models.}

The truth conditions for these models are then:
\begin{te}
$V_M(R^n_i(a_1,...,a_n)) = T \text{ iff } \alpha(R^n_i(a_1,...,a_n)) = T$
\end{te}
\begin{te}
  $V_M(\Sigma xA) = T$ if and only if there is a parameter $b$ in some
  parametric extension $\alpha' \in B$ of $M$ such that
  $V_{<\alpha', B>}(A[b/x]) = T$.\\
  \label{ext}
\end{te} 

The previous two criticism do not hold for the above semantics.  
In the case of first criticism, in \cite{bonevac84} and \cite{dunn}, proofs are given for the 
completeness and compactness of first-order logic with substitutional 
quantifiers similarly defined.
In the case of the second criticism, note that while the set of
constants will always be countable, extended or not, $B$ could be an
uncountableclass of permissible extensions. Thus, there would be an
uncountable number of names that could possibly be added to $C$ but
only a countable number that could ever be substituted into any
particular formula.  The issue is now not whether a name statically
exists in a language for an object $o$, but whether $o$ is
\emph{nameable} and it is unclear what it would even mean for an
object to be unnameable.\otodo[caption={Voldemort?}]{Voldemort = he
  who shall not be named}

\reorganize{Move me to the next section?}
It remains unclear how these semantics are to be understood
in relation to the original objectual quantifier and natural language.
More generally, it has not been illustrated that the substitutional
quantifier is even philosophically intelligible.  The next section
addresses these concerns.

%%% Local Variables: 
%%% mode: latex
%%% TeX-master: "substitutionalquantification"
%%% End: 