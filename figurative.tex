Following a brief argument in Yablo \cite{yablo} and Melia
\cite{melia}, I claim that even in scientific discourse such true
figurative language exists and that this entails the intractability of
ontological questions. To note this is not to critique scientific
practice, but to recognize certain objective methodological and
linguistic limitations, and the often predictive or operationally
descriptive goal of scientific theories. It should be a positive and
intuitive result that such figurative discourse need not be
ontologically binding.

That metaphor, in constructing an analogy between two concrete
domains, can be descriptive should not be controversial.  Such
figurative speech allows for greater semantic productivity in
articulating a way the world is. Yablo presents three such classes of
figurative speech that he claims are to be found even in our best
theories of the world: representationally essential metaphors,
expressing a way the world is that would otherwise be unrepresentable;
presentationally essential metaphors, emphasizing important aspects of
a theory; and procedurally essential metaphors, possessing no definite
content but that perhaps show possible paths for future research.

Of these, representationally essential metaphors are most important to
the present attack on the Quinean criteria of ontological commitment
as it is unclear whether she must admit the latter two kinds of
metaphor into her best theory of the world.  Yablo and Melia argue
that (g) ``the average star has 2.4 planets'' is an example of a
representationally essential metaphor.  While stars are presumed to be
in even the most stringent nominalist's ontology, the \emph{average
  star} is a notoriously abstract entity that is unfortunately useful
in describing general properties of stars.  Yet, even if scientists
have great evidence that the average star has 2.4 planets, it is
impossible that they could with any accuracy provide the exact number
of stars and planets in the universe. Our best astronomical theory,
then, certainly ought not be taken as committed to such a precise
count. A possible paraphrase would be ``the number of planets divided
by the number of stars is 2.4''.  Yet this sentence quantifies over
numbers and thus, says the Quinean ontologist, commits our best theory
to their existence.  We know, though, that there is a definite count
of planets in the universe but that such a count is inaccessible to
scientists. Thus the \emph{best} theory would contain a conjunctive
formula depicting the existence of n stars such as ``$\exists
x_1Star(x_1) \& \exists x_2Star(x_2) \& \ldots \& \exists
x_nStar(x_n)$'' and a similar sentence requiring the existence of the
m planets. The best theory, then, does not need to be committed to
numbers or average stars.  The only other choice for our best theory
is the infinite disjunctive phrase ``there are 12 planets and 5 stars
or 24 planets and 10 stars or\ldots'' which, being infinite, is
inexpressible in English.  

Therefore, supposing the Quinean criteria of ontological commitment,
it seems that \emph{our} best theory of astronomy is committed to
either numbers or the average star yet we know that \emph{the} best
theory need not be. Further, (g) is an example of a figurative
sentence that, if it or a similar statement was in a scientific theory
as seems likely, could neither be paraphrased away to avoid
ontological commitment or expected to disappear in favor a literal
statement in a future iteration of the theory.

The Quinean is thus faced with an unfortunate dilemma. (a) must either
be taken as literal or not.  If it is supposed to be literal, then our
best theory of the world demands that numbers or average stars exist
even though scientists know that such entities ought not be required
from this statement alone. If the Quinean rejects the existence of
numbers or average stars then (a) must be understood to be false no
matter how much evidence scientists have about planets and stars.  If
the statement is taken to be figurative, then it remains for the
Quinean to decide whether this statement should be removed from our
best theory of the world or whether ontology-free statements ought to
be accepted into scientific theories.  In the latter case, the Quinean
must provide an account of the meaning of and role played by such
figurative scientific sentences. If we allow our scientific theories
to be expressed in a formalism that includes the substitutional
quantifier, though, we can account for the truth of either (g) or its
explicitly mathematical paraphrase without having to countenance
numbers or average stars in the world.  

Briefly consider, too, the role of models in science.  By models we
will, following Hesse, take the broad understanding that they are
``any system, whether buildable, picturable, imaginable or none of
these which has the characteristic of making a theory predictive''
\cite[p.21]{hesse}.  For examples consider the billiard ball model of
a gas, the wave equation, a scale model of a bridge, or a computer
program simulating an arbitrary physical system.  I take it to be
uncontroversial that predictive modeling plays a major role in
scientific practice.  What is important to the present argument is
their often explicit disconnect from the world.  That is, what is
common to many models is that they cannot be taken to literally
represent the system they are analogous to.  For example, in the
billiard ball model of a gas, billiard balls possess properties such
as color that gas molecules do not. Consider, too, explicit
idealizations such as the frictionless plane which could not be.  All
of these models can successfully predict the behavior of a system and
perhaps hint at its ontological structure, but cannot be taken to
literally picture a way that the world is.

One would hope that deductions and predictive results from such models
could be included within our best scientific theories of the world,
but to take such derivations as ontologically committed to whatever
entities or structure exists in the model, such as the Quinean would
seemingly require, would be ludicrous given their principally
predictive function and often acknowledged structural difference from
how the world could be.  

This argument is certainly not conclusive, but it seems clearly
desirable that some statements about predictive models that are
acknowledged to differ from the world should be contained within our
best theories of the world and that they ought not be ontologically
binding.  That substitutional semantics are capable of accounting for
such non-ontological claims and their consistence with a model,
suggesting their similarity to talk of fictions like Pegasus, suggests
that the Quinean must also account for the ontological status of
scientific models and their relation to theories to not be at a
disadvantage.

I take the above two arguments to have demonstrated the feasibility of
such figurative language existing within our best theories of the
world.

%%% Local Variables: 
%%% mode: latex
%%% TeX-master: "substitutionalquantification"
%%% End: 
