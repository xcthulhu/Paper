\epigraph{
  It seems to me very curious that language\ldots should have grown up
  just as if it were expressly designed to mislead philosophers.}%
  {G. E. Moore \cite[p.217]{moore}}

%\begin{doublespace}

  As Quine has characterized it, the central question of ontology is
  ``what is there?"; its solution, ``everything"
  \cite[p.1]{quine}. This glib remark foreshadows the equivocation
  Quine would make between the formal conception of the logical
  quantifier and being.  It is an equivalence characterized by his
  criteria of ontological commitment: ``to be is to be the value of a
  variable'' \cite[p.15]{quine}\rewrite[caption={s/criteria/criterium/g?}]{There's only one
    \emph{criterium} being discussed in this entire paper; s/criteria/criterium/g?}.  Corresponding to this
  thesis is an appropriate methodology: ontological disagreements are
  to be resolved as semantic disputes over the implications of a
  theory's constituent statements. 
  \rewrite[caption={Comment about run-on sentences}]{\textbf{This paper is really good}, but you have
    run on sentences. This sentence here is a little long but I don't
    see how to change it.  I'm going to try to break them up when I
    see them, and you can review and decide if my rewording sounds
    strange/bad.}
  An interlocutor is taken to be
  ontologically committed to an entity $x$ if a sentence of the form
  ``there exists an $x$''---a sentence whose truth is supposed to
  demand some \emph{thing}, $x$---can be deduced from her disputed
  theory.  The Quinean argument schema is thus as follows:

\begin{te}
  Suppose theory $T$ is true.\\
  $T$ logically entails a sentence $S$ of the form `$\exists xFx$'.\\
  $S$ is true just in case there exists an $x$ such that it $F$s.\\
  Therefore there are $F$s.\\
\end{te}\label{arg1}

The Quinean meta-ontology has come to define contemporary ontological
disputes in the analytic tradition.  Yet this notion of commitment is
dependent upon a referential semantics for the existential quantifier
of first-order logic.  Consider the transition from line two to three
in (\ref{arg1}). This move is valid only if truth is model
theoretically defined in terms of satisfaction. 
% That is, it is only coherent to claim that the 
The truth of a logical formula such as
`$\exists xFx$' metaphysically commits one to the existence of an $F$
if this sentence's truth is dependent upon the presence of an $x$ that
$F$s in one's domain of discourse.  This referential interpretation of
quantification is by no means necessary, though, as there exists 
\otodo[caption={First order intuitionistic logic?}]{First order intuitionistic logic also has that $\vdash \exists x F x \iff \vdash
F c$ for some term $c$; I know this paper is geared towards semantics
however I think you should note this intuitionistic, proof theoretic
approach somewhere, especially if Dutch people are going to look at this}
an alternative substitutional quantifier, popularized by Saul Kripke
\cite{kripke} and Ruth Barcan Marcus \cite{marcus72}.  The
substitutional quantifier defines the truth of an existentially
quantified phrase in terms of substitution by logical constants or
names: `$\exists xFx$' is true in a model just in case there is a
substitution instance $c$ such that $Fc$ is true.

Natural language regularly referers to figurative and fictional
entities without clear distinction.  I argue that even rigorous
scientific theories mince real and figurative discourse.  This is seen
to be problematic for the Quinean criteria under the classical
semantics for the quantifier.  I conclude by arguing that Quine's
criteria may be salvaged by adopting the substitutional
quantifier.\ftfy[caption={FTFY: rewrote entire paragraph}]{This paragraph is important, so I rewrote it
  because it wasn't clear.  In the original paragraph, I got the sense
that you were rejecting Quine's criteria, where in reality you are
trying to salvage it by adopting alternate semantics for quantification.}

% The mere existence and prima facie formal validity of such a
% quantifier is insufficient to refute the Quinean project.  I contend,
% though, that the substitutional quantifier more intelligibly
% represents aspects of natural language dealing with figurative and
% fictional discourse.  I conclude that such figurative discourse could
% reasonably exist in our scientific theories of the world and that
% without a clear account of the distinction between figurative and
% literal uses of language the Quinean criteria of ontological
% commitment must fail.

%%% Local Variables: 
%%% mode: latex
%%% TeX-master: "substitutionalquantification"
%%% End: 
