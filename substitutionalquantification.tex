\documentclass[12pt,a4paper]{article}

\usepackage{ifpdf, amsmath, graphicx, amssymb, setspace}

% Geometry; change to suite preference
%\usepackage[a4paper,margin=.9in]{geometry}
\usepackage[a4paper,margin=.5in,right=2in]{geometry}
\setlength{\leftmargin}{2in}

% Spacing; don't use \begin{doublespace} with todonotes ~Matt
\usepackage{setspace}
\doublespacing

%\usepackage[authoryear]{natbib}

% Todo notes
\usepackage[colorlinks]{hyperref}
\usepackage[colorinlistoftodos,textwidth=1.75in]{todonotes}
\newcommand{\otodo}[2][]
{\todo[caption={#2}, #1]{\renewcommand{\baselinestretch}{1}\selectfont#2\par}}
\newcommand{\rewrite}[2][]
{\todo[caption={#2}, color=green!40,
  #1]{\renewcommand{\baselinestretch}{1}\selectfont#2\par}}
\newcommand{\ftfy}[2][]
{\todo[caption={FTFY: #2}, color=yellow,
  #1]{\renewcommand{\baselinestretch}{1}\selectfont#2\par}}
\newcommand{\delete}[2][]
{\todo[caption={#2}, color=black!40,
  #1]{\renewcommand{\baselinestretch}{1}\selectfont#2\par}}
\newcommand{\reorganize}[2][]
{\todo[caption={#2}, color=blue!75, #1]{\renewcommand{\baselinestretch}{1}\selectfont#2\par}}
\newcommand{\error}[2][]
{\todo[caption={#2}, color=red!40, #1]{\renewcommand{\baselinestretch}{1}\selectfont#2\par}}
\newcommand{\intodo}[2][]
{\todo[caption={#2}, inline, #1]{\renewcommand{\baselinestretch}{1}\selectfont#2\par}}


\usepackage{epigraph}
%\bibliographystyle{plainnat}
\bibliographystyle{plain}

\newenvironment{te}
{\begin{singlespace}
\begin{equation}
\begin{minipage}[t]{0.8\linewidth}}
{\end{minipage}
\end{equation}
\end{singlespace}
\ignorespacesafterend}

% Counters for automatically numbering counter-examples
\newcounter{counterex}
\setcounter{counterex}{0}
\newcommand{\cex}{\refstepcounter{counterex}\alph{counterex}}

\begin{document}
\listoftodos\pagebreak
\intodo{Are you sure that you want to put the date here?}
\begin{minipage}{0.5\textwidth}
\begin{flushleft}
\large{To Be The Value Of A Variable Is Not Always To Be.}\\
\end{flushleft}
\end{minipage}
\begin{minipage}{0.5\textwidth}
\begin{flushright}
Charles Francis \\
Phil 414 \\
May 8th, 2008
\end{flushright}
\end{minipage}

\section{Introduction}
\epigraph{
  It seems to me very curious that language\ldots should have grown up
  just as if it were expressly designed to mislead philosophers.}%
  {G. E. Moore \cite[p.217]{moore}}

%\begin{doublespace}

  As Quine has characterized it, the central question of ontology is
  ``what is there?"; its solution, ``everything"
  \cite[p.1]{quine}. This glib remark foreshadows the equivocation
  Quine would make between the formal conception of the logical
  quantifier and being.  It is an equivalence characterized by his
  criteria of ontological commitment: ``to be is to be the value of a
  variable'' \cite[p.15]{quine}\rewrite[caption={s/criteria/criterium/g?}]{There's only one
    \emph{criterium} being discussed in this entire paper; s/criteria/criterium/g?}.  Corresponding to this
  thesis is an appropriate methodology: ontological disagreements are
  to be resolved as semantic disputes over the implications of a
  theory's constituent statements. 
  \rewrite[caption={Comment about run-on sentences}]{\textbf{This paper is really good}, but you have
    run on sentences. This sentence here is a little long but I don't
    see how to change it.  I'm going to try to break them up when I
    see them, and you can review and decide if my rewording sounds
    strange/bad.}
  An interlocutor is taken to be
  ontologically committed to an entity $x$ if a sentence of the form
  ``there exists an $x$''---a sentence whose truth is supposed to
  demand some \emph{thing}, $x$---can be deduced from her disputed
  theory.  The Quinean argument schema is thus as follows:

\begin{te}
  Suppose theory $T$ is true.\\
  $T$ logically entails a sentence $S$ of the form `$\exists xFx$'.\\
  $S$ is true just in case there exists an $x$ such that it $F$s.\\
  Therefore there are $F$s.\\
\end{te}\label{arg1}

The Quinean meta-ontology has come to define contemporary ontological
disputes in the analytic tradition.  Yet this notion of commitment is
dependent upon a referential semantics for the existential quantifier
of first-order logic.  Consider the transition from line two to three
in (\ref{arg1}). This move is valid only if truth is model
theoretically defined in terms of satisfaction. 
% That is, it is only coherent to claim that the 
The truth of a logical formula such as
`$\exists xFx$' metaphysically commits one to the existence of an $F$
if this sentence's truth is dependent upon the presence of an $x$ that
$F$s in one's domain of discourse.  This referential interpretation of
quantification is by no means necessary, though, as there exists 
\otodo[caption={First order intuitionistic logic?}]{First order intuitionistic logic also has that $\vdash \exists x F x \iff \vdash
F c$ for some term $c$; I know this paper is geared towards semantics
however I think you should note this intuitionistic, proof theoretic
approach somewhere, especially if Dutch people are going to look at this}
an alternative substitutional quantifier, popularized by Saul Kripke
\cite{kripke} and Ruth Barcan Marcus \cite{marcus72}.  The
substitutional quantifier defines the truth of an existentially
quantified phrase in terms of substitution by logical constants or
names: `$\exists xFx$' is true in a model just in case there is a
substitution instance $c$ such that $Fc$ is true.

Natural language regularly referers to figurative and fictional
entities without clear distinction.  I argue that even rigorous
scientific theories mince real and figurative discourse.  This is seen
to be problematic for the Quinean criteria under the classical
semantics for the quantifier.  I conclude by arguing that Quine's
criteria may be salvaged by adopting the substitutional
quantifier.\ftfy[caption={FTFY: rewrote entire paragraph}]{This paragraph is important, so I rewrote it
  because it wasn't clear.  In the original paragraph, I got the sense
that you were rejecting Quine's criteria, where in reality you are
trying to salvage it by adopting alternate semantics for quantification.}

% The mere existence and prima facie formal validity of such a
% quantifier is insufficient to refute the Quinean project.  I contend,
% though, that the substitutional quantifier more intelligibly
% represents aspects of natural language dealing with figurative and
% fictional discourse.  I conclude that such figurative discourse could
% reasonably exist in our scientific theories of the world and that
% without a clear account of the distinction between figurative and
% literal uses of language the Quinean criteria of ontological
% commitment must fail.


\section{The Quinean Criteria and its Discontents}
\intodo[caption={Rewrite section title name?}]{This section presents
  Quine's traditional interpretation of his criterium and
  counter-examples refuting that perspective.  But this paper doesn't actually
  refute Quine's criterium (ie, ``To be is to be the value of a
  variable''), because ultimately you will salvage this idea with
substituional quantifiers.}

In this section I present Quine's traditional interpretation of his
criteria for ontological commitment, and provide examples of sentences
it has difficulty analyzing.\ftfy{I added an introductory sentence to
  this section summarizing what you do in it.  Maybe elaborate?}  
\intodo{After presenting the counter examples in this section, and the
flaws in Quines' analysis using traditional model theory, you might
want to return to the examples in this section to show how
the substitutional quantifier provides the appropriate semantics.}
\otodo[caption={Theories = collection of sentences?  What about
  deduction?}]{Aren't theories closed under deduction?  You talk about
  ``deriving'' things from them on the next page.  Also, ``theories''
  are a proof theoretic concept; your paper is about semantics, however.
  It may be prudent to caste your discussion in terms in of
  statements ``holding'' under a theory rather than being derived.} 
Quine's criteria of ontological commitment %will be understood as a
is a methodological claim.  It asserts the beings that must exist as a
consequence of a theory being true.  
A theory will be understood as a collection of
sentences. There are objections that can be raised against this
representation.  For example, one ought not suppose that the theory of
relativity is tied to any particular sentential formulation. Such
arguments will be ignored in the present work. Perhaps theories are in
the end more general entities, but this question is largely irrelevant
to the present concern with understanding whether it is reasonable to
logically question the ontological commitments of the sentences to
which an individual adheres.
\delete[caption={Delete this whole paragraph?}]{Defending a standard, 
logic based understanding of theories like this is not very relevant
to your paper.  Delete this whole paragraph?}

Quine claims the following as ``a standard whereby to decide what
ontology a given theory or form of discourse is committed to'':
\begin{te}
A theory is committed to those and only those entities to which the
bound variables of the theory must be capable of referring in order
that the affirmations in the theory be true \cite[p.13]{quine}.\\
\label{ontcommit}
\end{te}

Quine takes sentences such as (\cex) ``there is a dog''\label{cex:dog} to be
paradigmatic cases of ontological commitment.  This is intuitively
plausible.  If someone posits 
%the audacious claim that 
\eqref{cex:dog}, 
%it seems natural to take 
they are ontologically bound to the existence of something which is a dog.
%the speaker to some thing: a dog. It is also evident that 
The standard semantics of the
existential quantifier can represent this sentence and its appropriate
truth conditions in terms of dogs as `$\exists xDog(x)$'. 

%It is not intuitively clear, though, that this 
Logical translation does not
always clarify the ontological commitments at play, however.
Quine's analysis does not deal with figurative speech elegantly.
%has contributed to or clarified (a)'s utterer's commitment to a dog. 
%It seems that the burden of ontological commitment already resided in
%my understanding of the concept expressed by `dog'.  
Consider the claim that (s) ``there is a bee in her bonnet''. 
Most English-speaking individuals
%, inferring the speaker's intent and 
%understanding the idiomatic proposition expressed by this sentence, 
would take this utterance to be a 
%perhaps true 
claim about some woman's obsessiveness.
They would not consider the literal
existence of a bee in
anyone's bonnet.  In this way (s) is interpreted differently than (a),
even though it has he same form.  
Quine %takes it for granted that intentionally
grants that figurative speech is not ontologically binding, 
%but also supposes that the fanciful utterance in question must be
%taken 
by dismissing such sentences as false\footnote{
  ``One way in which a man may fail to share the ontological
  commitments of his discourse is \ldots by taking an attitude of
  frivolity.  The parent who tells the Cinderella story is no more
  committed to admitting a fairy godmother and a pumpkin coach into
  his own ontology than to admitting the story as true''
  \cite[p.103]{quine}.}. This is heavy handed and awkward.  
It is intuitively possible that (s) could communicate a
true proposition, while remaining ontologically unburdened with
anything but the existance of some girl.  This contradicts Quine's interpretation of his
ontological criteria. \ftfy{I heavily editted this, because it
  used a lot of indirect language}

In the above cases ontological commitment 
%appears to be largely
is determined by the intentions of a speaker and the meaning of the
phrases in question.  \delete[caption={Delete paragraph with run-on sentence?}]{This is a rather long run-one sentence in a
  paragraph that isn't doing much work.  Delete all of this?} 
Though there is no reason yet to suppose that
intentionally figurative language exists in one's best theory of the
world, it demonstrates a certain intuition---that language is used
in many different ways and first-order logic is incapable of
distinguishing between them---that could incline one towards a
rejection of the Quinean criteria of ontological commitment.
\intodo[caption={\emph{But you don't really reject Quine's
    criterium!}}]{\emph{But you don't really reject Quine's
    criterium!} 
You just reject Quine's traditional first order logic
  interpretation of his criterium.}

\otodo[caption={Stay away from worrying about ``derivation'' 
in a semantics paper}]{This paragraph begins by assuming theories involve derivation;
  shouldn't you be explicit about this?  Why not just say ``Suppose
  that if `there is an $x$' \textbf{holds} in a given theory then
  \ldots''?  That way you can stick to semantics which is all this
  paper is about.
  Also, avoid unecessary words like ``though''.}
Let us suppose, though, that if one can derive a literal statement of
the form
\reorganize{This is a nice, pithy way of summarizing Quine's
  interpretation of his ontological criterium. Why not say this at the
  beginning of the section, and then proceed to hammer it with your counter-examples?}
``there is an $x$'' from a given theory then the theory is committed to an appropriate $x$.
As indicated in (\ref{arg1}), the
typical manner of deriving such a statement when it is not explicitly
present in a theory is through translation into a first-order language
with standard semantics.  That is, one takes a claim such as (b)
``Pegasus is a winged horse'' and translates it into a closed formula
of first-order logic such as `$WingedHorse(pegasus)$' from which, via
existential generalization, one can derive the quantified expression
`$\exists xWingedHorse(x)$' from which it follows, based on the
Quinean understanding of the existential quantifier, that (c) ``there
exists a winged horse.''\rewrite{This paragraph is very confusing,
  along with the next paragraph.  Try to make it clear that 
``Pegasus is a winged horse'' is yet another counter-example, which
exposes the problem with the traditional Quinian criterium as a
problem with the semantics of first order logic. Also, be up front
that Quine/First Order Logic are overly dimissive of (c), 
just like the bee-bonnet sentence.}

Yet it seems evident that, while (b) is true, (c) is false.  There are
two standard responses.
\reorganize{You should be up-front that you
  are not entertaining Meinongian ontologies. Put this up at the
  beginning of the section or in the introduction.}
  First, one can follow Meinong and accept
abstract or non-existent entities into their domain of discourse and
thereby the world.  For those with nominalist leanings, this move is
unintelligible.  Second, one can follow Quine and reject the truth of
(b) on the grounds that no entity exists that satisfies `$\exists
xWingedHorse(x)$' and hence both (b) and (c) must be strictly speaking
false.  If one refuses to countenance abstract objects but insists
that (b) is true and (c) is false, some step of the preceding
derivation must be rejected.

Consider, too, a sentence like Marcus' (d) ``there is a statue of
Venus in the Louvre'' \cite{marcus72}.  For the nominalist, (d) must
be paraphrased to ``there is a statue named `Venus' in the Louvre"
which suffers from a lack of inferences I take to be valid (``there is
a statue of a Roman deity in the Louvre'') and a distortion of the
proposition seemingly expressed by (d): that there is a statue
\emph{of} Venus in the Louvre.

I take the denial of these sentences' truth to be quite unsatisfactory
in as much as it seems to clearly indicate the omission of facts I
find to be obvious.  The acceptance of abstract entities of whatever
caliber, though, is equally inexplicable.  No matter one's account of
such non-existent objects, this notion will constitute a more
significant conflict with common sense than an appeal to the cultural
(or fictional) conventions one employs and reasons about in day-to-day
discourse. Accounting for the truth of sentences like (b) and (d) as
wholly conventional (and hence the ontological demands of such
sentences as nonexistent) is significantly more intuitive than the
alternative objectual interpretation.  That is to say, I take the
specification of a domain $D$ for such discourse, especially in the
murky conceptual territories in which ontological disputes arise, to
be more mysterious than the acceptance of a fictional or conventional
provision of truth values for sentences.

This is not a move towards ontological relativity (that is, I do not
aim to advocate that what there is is a matter of convention), but
rather the claim that the truth of a sentence need not indicate
anything about the world but convention or, rather, that there is a
distinction between a theory of truth and the theory of reference one
needs to speak of ontological commitment.  The substitutional
quantifier can help to formalize the ambiguities at play here.


\section{The Substitutional Quantifier} 

Let a language $L$ consist of a countable set $V_L$ of individual
variables represented by `$x_1$', `$x_2$', \ldots, `$x_n$'; a
countable set $C_L$ of constants, represented by `$a_1$', `$a_2$',
\ldots, `$a_n$'; a finite or countable set $P_L$ of n-ary predicates for
each n, denoted `$R^n_1$', `$R^n_2$', \ldots, `$R^n_m$'; and the
logical symbols `$\neg$', `$\&$', `$\exists$', `$($', and `$)$'.

Recall the standard semantics for first-order logic.  A model $M$ of
$L$ is an ordered pair $<D,\phi>$ where $D$ is a non-empty set
constituting the domain of $M$ and $\phi$ is a mapping from elements
of $C_L$ to $D$ and from n-ary elements of $P_L$ to the powerset of $D^n$,
the set of all sets of n-membered subsets of $D$.  Let $V_M$ be
recursively defined as a valuation function on $M$ such that:

\begin{te}
  $V_M(R^n_i(b_1,\ldots ,b_n)) = T$ iff $\{ \phi(b_1),\ldots ,\phi(b_n) \} \in
  \phi(R^n_i)$
\label{existclause}
\end{te}

\begin{te}
  $V_M(\neg A) = T$ iff $V_M(A) = F$
\end{te}

\begin{te}
  $V_M(A \& B) = T$ iff $V_M(A) = V_M(B) = T$
\end{te}

\begin{te}
  $V_M(\exists xA) = T$ iff $V_{M'}(A[b/x]) = T$ for some
  parameter $b$ and some $M'$ differing from $M$ in at most what it
  assigns to $b$.\\
\label{quantifier}
\end{te}

It is (\ref{existclause}) that shows the referential nature of these
semantics: `$R(a,b)$' is true in a model only if the set containing the
extension of `$a$' and the extension of `$b$', a set containing objects
from the domain $D$, is a member of the extension of
`$R$'. (\ref{quantifier}) in turn tells us that `$\exists xF(x)$' is true
just in case there is some element $a \in D$ such that if $\phi(x) =
{a}$, $F(x)$ is true.  The truth of a quantified sentence is therefore
dependent on its satisfaction by objects in the domain of discourse $D$.

A naive account of substitutional semantics for first-order logic
requires several modifications to the above. So as to avoid notational
confusion, we will use the symbol $\Sigma$ to denote the
substitutional equivalent of the existential quantifier.  The formal
representation of naive substitutional quantification constitutes the
following replacements for (\ref{existclause}) and (\ref{quantifier}):

\begin{te} 
$V_M(R^n_i(b_1,\ldots ,b_n)) = T$ iff $M(R^n_i(b_1,\ldots ,b_n)) = T$
\end{te}

\begin{te}
$V_M(\Sigma xA) = T$ iff $V_{M'}(A[b/x]) = T$ for some parameter $b
\in C_L$.\\
\end{te} 

For a substitutional language $L$, then, a model $M$ is defined
merely as a mapping from the closed, atomic formula of $L$ to
$\{T,F\}$.  Note that this posits the truth of an existentially
quantified formula `$\Sigma xF(x)$' as dependent only on a substitution
instance $a$ in $L$ such that $M(F(a)) = T$; there is no talk of a
domain $D$.

This naive semantics, though, suffers from several notable
deficiencies.  First, the logic so described is neither compact nor
strongly complete.  If the substitutional quantifier is to be
supported for its pragmatic superiority to the objectual quantifier,
the former ought to share most of the latter's familiar logical
properties.

Second, such an interpretation is limited only to those terms
contained within the substitution class of parameters $C$ and thus,
since it cannot account for objects not already named in a language,
cannot account for much of natural discourse.  This is to say, the
above semantics can only discuss domains that are countable and named.
While we are interested in this quantifier for its ability to move us
past logical questions of ontological obligation, it must be capable
of representing discourse that makes recourse to uncountable domains,
for example science and mathematics. That is, the substitutional
quantifier must be at least as expressive as the standard existential
quantifier.

Marcus \cite{marcus95} suggests that, if this second objection is legitimate,
first-order logic with objectual quantification does not fare much
better given the Lowenheim-Skolem theorem.  Recall that if a theory
$T$ in a first-order language is satisfiable by an uncountable domain
then there exists a model $M'$ with a countable domain $D'$ that also
satisfies the sentences of $T$.  This result applies to the
first-order theory of real analysis that one would assume to be
satisfiable only by a domain with cardinality at least equal to that
of the real numbers. Consider now this model $M'$.  Since $D'$ is
enumerable, a name can be provided for every object contained therein.
Therefore, if we let the parameters of a substitutional language $L$
be the names of objects in $D'$, we can construct a substitutional
model for $L$ according to $M'$ that satisfies $T$.

As this response only addresses one of the complaints levied against
substitutional quantification, an alternative semantics provided by
Bonevac \cite{bonevac84}, accounting for both objections through what
he labels parametric substitutional semantics, is necessary.  A
parametric extension of a language $L$ is a language $L'$ defined to
be identical to $L$ except that $C_L \subset C_{L'}$ and $C_{L'}$ is
enumerable.  A model $M'$ for a language $L'$ is said to
parametrically extend a model $M$ for a language $L$ just in case $L'$
parametrically extends $L$ and for all atomic sentences $s$ of $L$,
$M'(s) = M(s)$; that is, $M'$ and $M$ agree on all of the atomic
sentences of $L$.  Let us denote such a restriction of a function $F$ to
a domain $D$ $F |_D$.

A naive parametric extensional semantics for a language $L$ would say
that a well-formed formula $f$ is true in a model $M$ just in case
there is a parametric extension of $M$ according to which $f$ is
true. Yet this would be quite problematic in that the language $L$ and
model $M$ could be modified without restriction; that is, such a
semantics seems to let one quantify over all logically possible
entities.  A manner of distinguishing between relevant and irrelevant
extensions of $L$ is necessary if this modified substitutional
quantifier is to be able to account for non-arbitrary scientific
discourse.

Let a parametric extensional model, then, consist of a standard
substitutional model coupled with a class of admissible parametric
extensions; that is, a parametric model $M$ of $L$ is an ordered pair
$<\alpha, B>$ such that $\alpha$ is a function from the atomic formula
of $L$ into truth values and $B$ is a class of standard models that
parametrically extend $\alpha$.  It should be noted here that each
parametric extension to $L$ has cardinality at most $\aleph_0$, but
that the class $B$ could contain uncountably many functions
designating admissible extensions.

The truth conditions for P are then:
\begin{te}
$V_M(R^n_i(a_1,...,a_n)) = T \text{ iff } \alpha(R^n_i(a_1,...,a_n)) = T$
\end{te}
\begin{te}
  $V_M(\Sigma xA) = T$ if and only if there is a parameter $b$ in some
  parametric extension $\alpha' \in B$ of $M$ such that
  $V_{<\alpha', B>}(A[b/x]) = T$.\\
  \label{ext}
\end{te} 

With this addition to the semantics of substitutional quantification
both of the preceding fears are assuaged.  In \cite{bonevac84} and
\cite{dunn} proofs are given for the completeness and compactness of
first-order logic with substitutional quantifiers similarly defined.
To address the second criticism, note that while the set of constants
will always be countable, extended or not, $B$ could be an uncountable
class of permissible extensions.  Thus, there would be an uncountable
number of names that could possibly be added to $C$ but only a
countable number that could ever be substituted into any particular
formula.  The issue is now not whether a name statically exists in a
language for an object $o$, but whether $o$ is \emph{nameable} and it
is unclear what it would even mean for an object to be unnameable.

So far, though, it remains unclear how this logic is to be understood
in relation to the original objectual quantifier and natural language
more generally, and whether it is philosophically intelligible.


\section{Philosophical Problems}

The most pressing concerns are how to account for the truth of atomic
sentences without recourse to an objectual domain, how to understand
the relation between the objectual and substitutional quantifiers, and
how to conceive the \emph{meaning} of a substitutionally quantified
phrase.

An obvious manner of providing the truth of atomic sentences is to
take the class of parameters, $C$, to contain properly referential
names and account for their truth or falsity by providing a domain of
discourse $D$ and a mapping $F$ from $C$ to $D$ and from n-ary
relations to the power set of $D^n$. A substitutional model $M$ can
now be defined such that $M(R^n_i(a_1, \ldots, a_n)) = T$ iff $\{F(a_1),
\ldots, F(a_n)\} \in F(R^n_i)$.  Here an atomic formula of a
substitutional language $L$ is true in a model if and only if it
contains a referent within the domain of discourse.  Such an
interpretation of the substitutional quantifier still defines the
truth of a first-order quantified sentence `$\Sigma xA$' in terms of
$A$'s truth when an atomic sentence is substituted for the variable
$x$ therein, but the truth of those substituents is dependent on
reference and hence any supposed ontological innocence is illusory.

This interpretation of a substitutional language is intelligible, but
clearly contrary to the metaphysical aims of the present paper.  It
does serve the purpose, though, of showing how substitutional
quantification can, in certain spheres of discourse, account for
ontological commitments.  That is, where the subject matter is taken
to be clearly defined (as is, one should note, almost never the case
in metaphysical disputes) and, as Marcus notes \cite{marcus72}, we can
be taken as already ontologically committed, then it is permissible to
see quantification as reinforcing such commitments by allowing the
substitution of referring names. This could be taken as a reversal of
the Quinean project on another front, with the bearer of ontological
burden reverting to the name.  For the adherent of the causal theory
of names, such a position might be appealing: ontological disputes
could become the consideration of a name's causal history in the
attempt to see whether it designates at all.

At the other extreme, a substitutional model could be understood
entirely without reference in terms of the conventions of a linguistic
community.  Such an interpretation could understand the substitutional
model to assign truth values to atomic sentences based on explicit
conventions of discourse or the specifications of a fictional corpus.
The parametric extension class, following Bonevac, can be seen to
contain different consistent frameworks or discourses and to detail
the language game relative to which we admit or reject candidates for
extension of our base assignment.  Bonevac uses such an
interpretation to account for the truth of mathematical statements
since, he argues, ``learning mathematics involves learning the rules
of mathematical discourse, and these rules allow existence assertions
on the basis of consistency alone" \cite[p.650]{bonevac84}.

Neither of these extremes seem \emph{philosophically} intelligible
given the assumption that we use language both to refer to the world
(such as should at some level involve reference) and to make
conventionally grounded statements (such as I cannot understand to
involve reference).  Kripke \cite{kripke} has demonstrated the
consistency of a logic that includes both objectual and substitutional
quantifiers that bind different kinds of variables. This formalism
would provide a means of reconciling these two intuitions.
Alternatively, the base parameters can be taken as standard
referential names, but extensions of this domain will be allowed
relative to convention or fiction.  For example, Greek mythology would
constitute an admissible extension that would allow one to make
certain true statements about Pegasus and Minotaurs without referring.

Either approach supports my underlying philosophical intuition: true
sentences can involve reference to the world or not and given some
sentence of natural language $s$ it is often ambiguous in which class
$s$ ought to fall.  In providing the logical form of $s$ with respect
to Kripke's formalism there would remain a substantive question of
what kind of variable, substitutional or objectual, to use in its
translation; in analyzing the logical form of a sentence according to
the second formalism, there would be, in metaphysical contexts,
the difficult question as to the status of a given name and whether it
refers.

Though both formalisms lose the austerity and elegance of the standard
semantics for first-order logic, I maintain that there is a genuine
problem of the truth of sentences such as those involving Pegasus that
is unresolved using the standard interpretation of first-order logic.
Logical simplicity does not appear an adequate compensation for this
loss.

Another objection has been raised by Van Inwagen \cite{inwagen} in
which he argues that the meaning of a substitutionally quantified
phrase is incomprehensible.  The proposition expressed by `$\exists
xDog(x)$' is clear: there exists a dog. Yet while the truth conditions
of the expression `$\Sigma xDog(x)$' are known, the \emph{meaning} of
this phrase is not unless it is the metalinguistic proposition that
there is some true substitution instance of `$Dog(x)$'.  Certainly this
is not the meaning of ``there are dogs'', though.  For example, one
could consent to ``there are dogs'' without holding beliefs about
sentences.

Scott Soames \cite[p.91]{soames} responds that if Van Inwagen's
complaint is legitimate, how are we to subsequently understand the
meaning of $A \& B$?  Since the truth of conjunction is defined in
terms of $A$'s being true and $B$'s being true, it seems to follow
that the meaning of `Dog(fido) \& Dog(lassy)' is a metalinguistic
proposition about the truth of `Dog(fido)' and `Dog(lassy)', but
this is surely not be the case.

This response hints at the fundamental problem of Van Inwagen's
critique: it seems to assume an identity between a theory of truth and
a theory of meaning. I have already claimed that sentences such as
``there are dogs'' can be taken as the clearest cases of ontological
commitment.  Thus, I do not find it problematic to claim that the
proposition expressed by (e) `$\Sigma xDog(x)$', if it is not already
a mistake to speak of the proposition expressed by a logical formula,
is the same as that expressed by (f) `$\exists xDog(x)$' which is the
same as that expressed by ``there are dogs''.  The difference remains
in how we account for the truth conditions of these sentences.  The
truth of (f) will be defined directly in terms of the dogs in the
world; the truth of (e) will be defined in terms of there being a name
$n$ such that `$Dog(n)$' is true and, as a contingent fact about the
meaning of `Dog', it will be the case that $n$ designates a dog.  In
following this line of argument, then, the only necessary constraints
on a theory of meaning are that meanings cannot be defined in terms of
reference and that a theory of meaning cannot be a theory of
truth. This goes against some traditional threads in the philosophy of
language but can perhaps find company with those who insist on an
external and conventional or normative account of meaning.

If no substantial technical difficulties are found, substitutional
quantification so defined provides a more coherent account of
fictional and conventional discourse than that provided by the use of
quantification over abstract entities, thereby resolving a non-trivial
philosophical problem. Some trade-offs for accepting this
philosophical use of the formalism are the constraints imposed on
one's theory of meaning and added logical complexity, but I contend
that this is a fair trade-off. 

Yet, that truths in everyday discourse can be grounded in convention
or fiction is relatively uncontroversial. The aspiring ontologist can
at this point still rebut that nothing demonstrated speaks to the
ontological commitments of the scientific discourse to be found in our
best theories of the world which, the ontologist hopes, does not
similarly make recourse to such non-committal figurative language.


\section{Figurative Science}
Following a brief argument in Yablo \cite{yablo} and Melia
\cite{melia}, I claim that even in scientific discourse such true
figurative language exists and that this entails the intractability of
ontological questions. To note this is not to critique scientific
practice, but to recognize certain objective methodological and
linguistic limitations, and the often predictive or operationally
descriptive goal of scientific theories. It should be a positive and
intuitive result that such figurative discourse need not be
ontologically binding.

That metaphor, in constructing an analogy between two concrete
domains, can be descriptive should not be controversial.  Such
figurative speech allows for greater semantic productivity in
articulating a way the world is. Yablo presents three such classes of
figurative speech that he claims are to be found even in our best
theories of the world: representationally essential metaphors,
expressing a way the world is that would otherwise be unrepresentable;
presentationally essential metaphors, emphasizing important aspects of
a theory; and procedurally essential metaphors, possessing no definite
content but that perhaps show possible paths for future research.

Of these, representationally essential metaphors are most important to
the present attack on the Quinean criteria of ontological commitment
as it is unclear whether she must admit the latter two kinds of
metaphor into her best theory of the world.  Yablo and Melia argue
that (g) ``the average star has 2.4 planets'' is an example of a
representationally essential metaphor.  While stars are presumed to be
in even the most stringent nominalist's ontology, the \emph{average
  star} is a notoriously abstract entity that is unfortunately useful
in describing general properties of stars.  Yet, even if scientists
have great evidence that the average star has 2.4 planets, it is
impossible that they could with any accuracy provide the exact number
of stars and planets in the universe. Our best astronomical theory,
then, certainly ought not be taken as committed to such a precise
count. A possible paraphrase would be ``the number of planets divided
by the number of stars is 2.4''.  Yet this sentence quantifies over
numbers and thus, says the Quinean ontologist, commits our best theory
to their existence.  We know, though, that there is a definite count
of planets in the universe but that such a count is inaccessible to
scientists. Thus the \emph{best} theory would contain a conjunctive
formula depicting the existence of n stars such as ``$\exists
x_1Star(x_1) \& \exists x_2Star(x_2) \& \ldots \& \exists
x_nStar(x_n)$'' and a similar sentence requiring the existence of the
m planets. The best theory, then, does not need to be committed to
numbers or average stars.  The only other choice for our best theory
is the infinite disjunctive phrase ``there are 12 planets and 5 stars
or 24 planets and 10 stars or\ldots'' which, being infinite, is
inexpressible in English.  

Therefore, supposing the Quinean criteria of ontological commitment,
it seems that \emph{our} best theory of astronomy is committed to
either numbers or the average star yet we know that \emph{the} best
theory need not be. Further, (g) is an example of a figurative
sentence that, if it or a similar statement was in a scientific theory
as seems likely, could neither be paraphrased away to avoid
ontological commitment or expected to disappear in favor a literal
statement in a future iteration of the theory.

The Quinean is thus faced with an unfortunate dilemma. (a) must either
be taken as literal or not.  If it is supposed to be literal, then our
best theory of the world demands that numbers or average stars exist
even though scientists know that such entities ought not be required
from this statement alone. If the Quinean rejects the existence of
numbers or average stars then (a) must be understood to be false no
matter how much evidence scientists have about planets and stars.  If
the statement is taken to be figurative, then it remains for the
Quinean to decide whether this statement should be removed from our
best theory of the world or whether ontology-free statements ought to
be accepted into scientific theories.  In the latter case, the Quinean
must provide an account of the meaning of and role played by such
figurative scientific sentences. If we allow our scientific theories
to be expressed in a formalism that includes the substitutional
quantifier, though, we can account for the truth of either (g) or its
explicitly mathematical paraphrase without having to countenance
numbers or average stars in the world.  

Briefly consider, too, the role of models in science.  By models we
will, following Hesse, take the broad understanding that they are
``any system, whether buildable, picturable, imaginable or none of
these which has the characteristic of making a theory predictive''
\cite[p.21]{hesse}.  For examples consider the billiard ball model of
a gas, the wave equation, a scale model of a bridge, or a computer
program simulating an arbitrary physical system.  I take it to be
uncontroversial that predictive modeling plays a major role in
scientific practice.  What is important to the present argument is
their often explicit disconnect from the world.  That is, what is
common to many models is that they cannot be taken to literally
represent the system they are analogous to.  For example, in the
billiard ball model of a gas, billiard balls possess properties such
as color that gas molecules do not. Consider, too, explicit
idealizations such as the frictionless plane which could not be.  All
of these models can successfully predict the behavior of a system and
perhaps hint at its ontological structure, but cannot be taken to
literally picture a way that the world is.

One would hope that deductions and predictive results from such models
could be included within our best scientific theories of the world,
but to take such derivations as ontologically committed to whatever
entities or structure exists in the model, such as the Quinean would
seemingly require, would be ludicrous given their principally
predictive function and often acknowledged structural difference from
how the world could be.  

This argument is certainly not conclusive, but it seems clearly
desirable that some statements about predictive models that are
acknowledged to differ from the world should be contained within our
best theories of the world and that they ought not be ontologically
binding.  That substitutional semantics are capable of accounting for
such non-ontological claims and their consistence with a model,
suggesting their similarity to talk of fictions like Pegasus, suggests
that the Quinean must also account for the ontological status of
scientific models and their relation to theories to not be at a
disadvantage.

I take the above two arguments to have demonstrated the feasibility of
such figurative language existing within our best theories of the
world.


\section{Conclusion.}

If nothing else, the present paper can be taken to show the utility of
the substitutional quantifier for the rigorous nominalist and to point
out several non-trivial difficulties for the Quinean criteria of
ontological commitment.  By conceding that fictional discourse is not
ontologically committing Quine has committed himself to being able to
distinguish between those aspects of our best theory of the world that
are put forth in a literal way and those that are not.  Much like Quine
earlier argued against the notion of analyticity, there seems to be no
clear criteria of determining whether an assertion is to be taken
literally or figuratively in difficult cases and hence no clear
criteria of understanding their ontological commitments.

Further, the distinction this argument speaks to, that between literal
and figurative or, more generally, referential and non-referential
sentences seems to be plausibly represented by the formalism
championed herein.  This conjoined with its ability to address the
earlier linguistic intuitions seems to constitute a substantive
argument in favor of its philosophical utility. The interpretation of
this formalism and its relation to natural language clearly needs more
development and exploration (especially in relation to the theory of
meaning), but it appears to be a promising research path for the
nominalist or quietist who desires to dissolve or displace ontological
disputes. Marcus \cite{marcus72} characterizes the existence of an
object as being shown by criteria of evidence and presuppositions of
identity.  This and similar programs seem much more sensible
approaches to ontological disputes if the latter are to continue at
all.

%\end{doublespace}

\newpage

\bibliography{substitutionalquantification}{}

\end{document}
